\section{Introduction}
\label{sec:introduction_cap2}

The PoliMi Space Agency wants to launch a Planetary Explorer Mission, to  perform  Earth Observation. This section carries out relevant orbital analysis and groundtrack estimation while also considering two perturbation models. A modified groundtrack was proposed for a repeating groundtrack, and two propagation methods are used to perform the analysis which are then compared. A comparison between the real data of a satellite and its analytical results obtained with the code model is also performed for model validity.


\subsection{Nominal Orbit}

From the provided orbital parameters this satellite heavily contains geosynchronous orbital characteristics. Hence, the altitude at perigee is chosen as 35786 km \cite{perigee_alt} - where it is possible to see the moon and J2 perturbation effect. $\Omega$ (right ascension of ascending node), $\omega$ (argument of perigee), and $f_0$ (initial true anomaly) are chosen arbitrarily for a simpler analysis.

\begin{table}[ht]
	\centering
	\label{tab:keplerian_elements}
	\begin{tabular}{|c|c|c|c|c|c|}
		\hline
		a [km] & e [-] & i [°] & $\Omega$ [°] & $\omega$ [°] & $h_p$ [km] \\
		\hline
		42159 & 0.0007 & 32.5934 & 0 & 85 & 35786 \\
		\hline
	\end{tabular}
	\caption{Keplerian elements of the orbit}
\end{table}

The unperturbed nominal orbit is propagated as below in the Earth-centered reference frame:

\cfig{nominal_orbit.eps}{Assigned orbit}{nominal_orbit}{0.85}