\section{Algorithms description}
\label{sec:algo_description}

The targeting problem previously defined can be seen as an optimization problem with three degrees of freedom (DOFs). Indeed, once the departing date and the two times of flight of the heliocentric legs are chosen, both the Lambert's arcs and the fly-by hyperbola are fully defined.  
Regarding the formulation of the Lambert's problem, it requires the knowledge of the initial and final position and also the imposed time of flight between them. For two Lambert's arcs it would be needed a total of six DOFs, but this quantities are dependant one to each other:
\begin{itemize}
    [wide,itemsep=3pt,topsep=3pt]
    \item the final position vector of the first arc corresponds to the initial position of the second one;
    \item the initial date for the departing on the second arc corresponds the arrival date on the first arc;
    \item fixing the first Lambert's arc, the final position and arrival date for the second Lambert's arc are related through the analytical ephemerides.
\end{itemize}

Once the two heliocentric legs are determined, the powered gravity assist follows as the geocentric velocity vectors are known.

The method implemented to solve the optimization problem is the \textbf{brute force algorithm} refined with the \textbf{gradient descent algorithm}. Then, to validate the results, the \textbf{brute force algorithm} has been used solely with a more dense search grid in a reasonable domain. 

\subsection{Refined brute force algorithm}
\label{subsec:brute_force_algo}

\begin{algorithm}
    \caption{Bruteforce algorithm} \label{alg:brute}
    \begin{algorithmic}
    \Require $T_{dep}, \Delta T_{1}, \Delta T_{2}$
    \State $\Delta V_{min} = 10^{10}$
    \For{$i$ in $T_{dep}$}  
        \For{$j$ in $\Delta T_{1}$} 
            \State Calculate $T_{fly-by}$, first Lambert's arc, Mars' velocity at $T_{dep}$ and  $\Delta V_1$
            \For{$k$ in $\Delta T_{2}$}
                \State Calculate $T_{arr}$, second Lambert's arc, Asteroid's velocity at $T_{arr}$, $\Delta V_2$, $\Delta V_3$, $\Delta V_{tot}$
                \If{$\Delta v_{tot} < \Delta v_{min}$ and $r_p > r_{Earth} + 500 \; km$}
                \State $\Delta v_{min} = \Delta v_{tot}$
                \State $T_{dep,min} = T_{dep} $;   $T_{fb,min}  = T_{fb}$;   $T_{arr,min} = T_{arr}$
                \EndIf
            \EndFor
        \EndFor
    \EndFor
\\
\State Minimize cost using \textit{fminunc} with initial guess  $\left(T_{dep,min};T_{fb,min};T_{arr,min} \right)$
\\
\Return $\Delta V_{min}$; $T_{dep,min}$; $T_{fb,min}$; $T_{arr,min}$
    \end{algorithmic}
\end{algorithm}

The presented \autoref{alg:brute} is reliable, yet  computationally demanding since the research of the minimum is performed through a triple-nested \textit{for} loop. The time of the execution highly depends on the refinement chosen for the three selected periods of time. From this considerations and also noticing that the function to minimize presents high irregularities, it is clear that narrow time windows can help in the overall research. Moreover, with this last observation, it is possible to use a less fine grid of research, since it is reasonable to think that the function will present less irregularities.

As a consequence, it was decided to perform a physical analysis (\autoref{sec:time_red}) in order to wisely reduce the time domains for departure, fly-by and arrival. The idea of \autoref{alg:brute} is that, once this reduction is performed, the brute-force search can be carried out on a small but refined grid. To speed up the process and refine the solution, \textit{fminunc} of \textit{MATLAB} was used. The selected initial guess is the outcome of the brute force research.

In order to completely validate the results, the robustness of the brute force algorithm has been exploited. Referring to \autoref{alg:brute}, \textit{fminunc} was removed after the for-loop search and the research grid was refined. Moreover, with this computation it is possible to validate the time reduction analyzed in \autoref{sec:time_red}.


