\section{Conclusion and results}
\label{sec:conclusion}
\subsection{heliocentric legs}
\label{subsec:heliocentric}
The two eliocentric legs associated with the solution computed through ref aloritmo, can be characterized through the keplerian elements, shown in \autoref{table:Orb_Par}:
\begin{table}[H]

    \centering
    \begin{tabular}{|c|c|c|c|c|c|c|c|c|c|}
    \hline
    Arc & Dep & Arr & $a  \ [AU]$ & $e \ [-]$ & $i \ [deg]$ & $\Omega \ [deg]$ & $\omega \ [deg]$ & $\theta_{dep} \ [deg]$ & $\theta_{arr} \ [deg]$ \\
    \hline
    M$\to$E & $3 \ June \ 2033$ & $10 \ Feb \ 2034$ & $1.1602$ & $0.2870$ & $1.18$ & $321.55$ & $105.97$ & $195.38$ & $74.03$ \\
    \hline
    E$\to$G & $10 \ Feb \ 2034$ & $27 \ March \ 2036$ & $1.9445$ & $0.5029$ & $4.17$ & $141.55$ & $339.91$ & $20.09$ & $231.37$ \\
    \hline
    \end{tabular}
    
    \caption{Heliocentric legs}
    \label{table:Orb_Par} 
\end{table}

\subsection{gravity assist}

\label{subsec:ga}
Known the incoming and outcoming heliocentric velocities of the spacecraft it was possible to completely characterize the 
fly by hyperbola in an Earth-centred ecliptic frame. $v_{\infty}^{-}$ and $v_{\infty}^{+}$, computed by substracting 
the velocity of the planet to the incoming and outcoming velocity of the Lambert's arcs,,will in general be different, so
the gravity assist had to be powered: an impulse $\Delta v_{3}=2.7715 \ 10^{-6}  \ Km/s $ was given at the common pericenter
of the two hyperbolic arcs. All the most relevant parameters are shown in table.

\cite{curtis_book}