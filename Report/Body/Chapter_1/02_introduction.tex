\section{Introduction}
\label{sec:introduction}

\subsection{Description of the problem}
\label{subsec:description}

The first part of the assignment aims at designing an interplanetary transfer from Mars to asteroid 1036 Ganymed exploiting a powered gravity assist on Earth. The problem is analyzed through the patched conics method, without considering the injection and arrival hyperbolae. The initial and final velocity vector of the satellite are assumed to be the same of the respective celestial body. The two heliocentric legs are calculated through the Lambert problem. The trajectory is selected with the only criteria of minimizing the cost of the mission, assessed through the total $\Delta V$. The latter is computed by summing different contributions:

\begin{equation}
    \Delta V_{tot}= \Delta V_1 + \Delta V_2 + \Delta V_3
\end{equation}

Where the three terms are defined as:

\begin{itemize}
    [wide,itemsep=3pt,topsep=3pt]
    \item $\Delta V_1$ related to the injection in the first heliocentric leg;
    \item $\Delta V_2$ related to the exit from the second heliocentric leg;
    \item $\Delta V_3$ related to the impulse given by the engine at pericentre of hyperbola fly-by;
\end{itemize}

All the manoeuvres are assumed to be impulsive, i.e. they change only the velocity vector of the spacecraft, mantaining invariated the position vector. Note that $\Delta V_{1}$ and $\Delta V_{2}$ are related to heliocentric velocities, while $\Delta V_{3}$ is calculated through relative geocentric velocities. 

For each manoeuvre the cost is computed as:

\begin{equation}
    \Delta V_i= \lVert \boldsymbol{V_{+,i}} - \boldsymbol{V_{-,i}} \rVert
\end{equation}

where $\boldsymbol{V_{-,i}}$ and $\boldsymbol{V_{+,i}}$ are the velocity vectors before and after the i-th manoeuvre respectively.

\subsection{Assigned data and constraints}
\label{subsec:data_constraints}

A few constraints were considered:

\begin{itemize}
    [wide,itemsep=3pt,topsep=3pt]
    \item earliest departure date $t_{min,dep} = \left[01/01/2028 \;\; 00:00:00\right]$ in Gregorian calendar
    \item latest arrival date $t_{max,arr} = \left[01/01/2058 \;\; 00:00:00\right]$ in Gregorian calendar
    \item time of flights of the two heliocentric arcs must be greater than the associated parabolic time
    \item minimum pericentre radius of the fly-by hyperbola $r_p = r_E + 500 \; km$
    \item single-revolution Lambert problem was considered
    \item reasonable total time of the mission
\end{itemize}

Note that the constraint on the pericentre radius of the fly-by is considered both for avoid impact on Earth and to prevent undesired atmospheric drag effects.